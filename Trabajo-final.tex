\documentclass[oneside]{article}
\usepackage{fancyhdr}
\usepackage{lastpage}

\pagestyle{fancy}
\fancyhf{}
\lhead{\textbf{ANALISTA UNIVERSIATARIO EN SISTEMAS \\ TALLER DE PROGRAMACION I}}
\rfoot{AUS - Taller de programación I}
\lfoot{Pagina \thepage{} de \pageref{LastPage}}

\begin{document}

	\begin{titlepage}
		\vspace*{\fill}		
		\centering
		{\huge \bfseries Trabajo final WORDLE\par}
		\vspace{1cm}
		{\large Andrés Guido Grillo\par}		
		\vfill
	\end{titlepage}
	
	\section*{Documentacion de ejercicio}
	\begin{itemize}
		\item Objetivo\hfill 2
		\item Práctica C\hfill ??
		\item Algoritmos y funciones importantes\hfill ??
		\item Compilacion y ejecucion \hfill ??
	\end{itemize}

	\newpage
	\thispagestyle{fancy}	

	\section*{Objetivo}
	El objeto del ejercicio es programar el juego \textbf{WORDLE}.\\
	El juego consiste en un tablero con espacio para poder poner un máximo de 5 letras en donde el jugador deberá intentar adivinar una palabra oculta escribiendo un intento en dicho tablero.\\
	Dependiendo de la posición individual de cada letra se le comunicara al jugador si dicha letra está bien posicionada, si pertenece a la palabra o si directamente la letra no está incluida en la palabra secreta. La forma en la que se le comunica al jugador que letras son correctas es de la manera siguiente:
	\begin{itemize}
	\item Si la letra ingresada está en la posición correcta debajo de dicha letra aparecerá el caracter "V"
	\item Si la letra ingresada pertenece a la palabra pero no está en la posición correcta, debajo de la letra aparecerá el caracter "-"
	\item Si la letra no pertenece a la palabra entonces aparecerá el caracter "\_"
	\end{itemize}
	El jugador tiene un máximo de 5 intentos por cada palabra y se le asignará un puntaje en función de la cantidad de intentos, la cantidad de letras acertadas y si se logró descubrir la palabra secreta o nó, siendo el mayor puntaje posible de obtener 10000(diez mil) puntos. La meta del juego es tratar de obtener la mayor cantidad de puntos.
	\newpage
	\thispagestyle{fancy}
\end{document}